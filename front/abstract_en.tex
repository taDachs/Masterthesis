% !TeX spellcheck = en_US
% !TeX encoding = UTF-8
% !TeX root = ../thesis.tex

\chapter*{Abstract}
\addcontentsline{toc}{chapter}{Abstract}

This thesis investigates how scalable off-policy reinforcement learning can be applied to real-time
locomotion tasks on legged robots. Building on the FastTD3 algorithm, the work extends the baseline
with (i) a fixed-proportion offline buffer for teacher data, (ii) a recurrent policy for temporal
fusion, and (iii) an egocentric vision pathway in the actor network. These additions address the
limited stability and sample efficiency often encountered when scaling off-policy methods to
high-dimensional, partially observable control problems.

Experiments were conducted in IsaacLab using a Unitree GO2 quadruped across diverse terrain
scenarios, including gaps, pits, and constrained spaces. Two teacher–student training strategies
were compared: a static offline buffer seeded with privileged rollouts, and a hybrid approach that
integrates a time-decaying behavior-cloning loss into online learning. The hybrid setup achieved the
most consistent convergence and highest task performance, outperforming both baseline and
offline-only methods.

The trained student policy was successfully deployed on the real robot using an Intel NUC for
on-device ONNX inference. The robot demonstrated robust crawling behavior under low obstacles,
confirming partial transfer of the learned policy to real-world conditions. The results highlight
that combining off-policy reinforcement learning with moderate imitation supervision and egocentric
perception can yield scalable and deployable locomotion controllers.
