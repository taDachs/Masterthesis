% !TeX spellcheck = en_US
% !TeX encoding = UTF-8
% !TeX root = ../thesis.tex

\chapter{Evaluation}\label{chap:evaluation}

This chapter evaluates the performance of the different learning approaches introduced in the
previous sections. The aim is to answer two central questions:  
(1)~which approach achieves the best overall performance on the terrain task, and  
(2)~how the addition of a behavior-cloning term influences the training dynamics and final results.

The baseline RL agent, the teacher (privileged PPO), pure behavior cloning, the hybrid method, and
the RLPD (teacher-buffer) variant are compared. The following sections present the quantitative
results and describe the observed training behavior.

\section{Overall Comparison of Approaches}\label{sec:overall-comparison}

Figure~\ref{fig:terrain-level-approaches} shows the terrain levels achieved during training for all
evaluated approaches. The teacher agent reaches the highest terrain levels and clearly outperforms
the remaining methods, which follows from its access to privileged state information that is
unavailable to the student policy.

Among the student-only approaches, pure behavior cloning and the hybrid method perform almost
identically, with behavior cloning showing a very small advantage. The hybrid method does not
meaningfully improve over behavior cloning in terms of terrain level, suggesting that behavior
cloning already extracts most of the information recoverable from partial observations. The partial
observability of the environment likely limits the achievable performance for both methods.

The baseline RL agent and the RLPD variant perform substantially worse and do not make meaningful
terrain progress. RLPD was also observed to be highly sensitive to the collected replay buffer: even
buffers recorded under the same seed produced noticeably different outcomes. This sensitivity,
combined with the low performance, indicates that RLPD does not reliably make use of the teacher
data in this setting.

\begin{figure}
  \centering
  \includegraphics[width=0.99\textwidth]{figures/evaluation/terrain_level.png}
  \caption{Terrain level performance across training for all evaluated approaches. The teacher
  (privileged PPO) reaches the highest levels, while behavior cloning and the hybrid method perform
  similarly and substantially better than the baseline and RLPD. Right: distribution of final
  terrain levels across runs.}
  \label{fig:terrain-level-approaches}
\end{figure}

\section{Effect of Behavior-Cloning Weight}\label{sec:lambda-effects}

Figure~\ref{fig:terrain-level-hybrid-lambda} illustrates the influence of the behavior-cloning
weight~$\lambda_{\text{max}}$ by showing terrain level, Q-loss, and actor-loss for several choices
of the weight. Lower values of~$\lambda_{\text{max}}$ reduce the achieved terrain level but improve
both Q-loss and actor-loss. Removing the decay term (\(\alpha = 0\)) results in the worst losses,
whereas runs with decay (\(\alpha = 3\)) show better behavior. Lower weights correspond to more
off-policy learning, but weights that are too small reduce the final terrain level by roughly one
level.

\begin{figure}
  \centering
  \includegraphics[width=0.99\textwidth]{figures/evaluation/hybrid-mode-all-plots.png}
  \caption{Terrain level (left), Q-loss (middle), and actor-loss (right) for different
  behavior-cloning weights~$\lambda_{\text{max}}$. Lower weights improve Q-loss and actor-loss but
  reduce the final terrain level. Runs without decay (\(\alpha = 0\)) show the worst losses. Very
  small weights reduce terrain performance by about one level.}
  \label{fig:terrain-level-hybrid-lambda}
\end{figure}

The effect of the behavior-cloning weight is also reflected in the mean reward
(Figure~\ref{fig:mean-reward-hybrid}). Higher values of~$\lambda_{\text{max}}$ or disabling the
decay term lead to the best rewards and outperform pure behavior cloning. Very small weights reduce
the reward. This indicates that off-policy training is taking place and allows the hybrid approach
to outperform pure behavior cloning when both are trained for the same number of environment steps.

\begin{figure}
  \centering
  \includegraphics[width=0.99\textwidth]{figures/evaluation/hybrid_mean_reward.png}
  \caption{Mean reward for pure behavior cloning and hybrid training with different
  behavior-cloning weights~$\lambda_{\text{max}}$. Higher weights and the no-decay setting achieve
  the best rewards and outperform pure behavior cloning. Very small weights reduce the reward.}
  \label{fig:mean-reward-hybrid}
\end{figure}

A related effect is visible in the gradient norms of the vision backbone
(Figure~\ref{fig:vision-grad-norm}). Higher values of~$\lambda_{\text{max}}$ produce stronger
gradients, indicating that the supervised loss provides a stronger training signal for the backbone.
Behavior cloning exhibits the largest gradients overall.

\begin{figure}
  \centering
  \includegraphics[width=0.99\textwidth]{figures/evaluation/vision_grad_norm.png}
  \caption{Gradient norm of the vision backbone for pure behavior cloning and different
  behavior-cloning weights~$\lambda_{\text{max}}$. Higher weights result in stronger gradients, and
  behavior cloning shows the largest gradient norms.}
  \label{fig:vision-grad-norm}
\end{figure}

\section{Vision Representations}\label{sec:vision-representations}

Figure~\ref{fig:vision-latent-pca} shows a PCA projection of the vision latent space for the
different approaches. Only the crawl task forms a separate cluster across all methods. The vision
encoder appears to be overfitted to this task, while all other tasks and obstacles collapse into the
same cluster. This behavior is also reflected in the agent’s performance: only the crawl task makes
clear use of the visual input, whereas the remaining tasks can be solved without relying on vision.

\begin{figure}
  \centering
  \includegraphics[width=0.85\textwidth]{figures/evaluation/vision_latent_space.png}
  \caption{PCA projection of the vision latent space for the different approaches. The crawl task
  forms a distinct cluster for all methods, while all other tasks overlap.}
  \label{fig:vision-latent-pca}
\end{figure}

\section{Performance on the Real Robot}\label{sec:real-robot}

The trained student policy was deployed on the Unitree GO2 robot for qualitative testing. Deployment
was conducted using the setup described in \autoref{sec:deployment}, with an Intel NUC handling
ONNX inference and depth image preprocessing. The policy was transferred without additional
fine-tuning in the real world, showing that domain randomization was sufficient for sim2real
transfer.

In real-world trials, the policy successfully detected low obstacles and executed crawling motions
underneath them, as shown in Figures~\ref{fig:obstacle} and~\ref{fig:crawl}. The robot lowered its
body posture and passed under obstacles with a clearance of approximately 27\,cm, while its height
with payload was around 45\,cm.

Performance on other obstacle types , such as step-ups or more irregular terrain , was reduced compared
to simulation. The robot doesn't seem to use its camera for detecting the obstacle, instead relying
on tactile perception to detect when to move it's legs up. The same behavior can be observed in the
simulation. However, the resulting motions are too violent, and lead to the robot crashing.

During flat-ground locomotion, the policy appeared less stable than the blind
variant, which may indicate partial overfitting to the crawling task. This observation is consistent
with the latent-space visualization in Figure~\ref{fig:vision-latent-pca}, where the crawl scene
forms a separate cluster.

\begin{figure}
  \centering
  \includegraphics[width=0.75\textwidth]{figures/evaluation/robot_obstacle.jpeg}
  \caption{Comparison between the robot height (45\,cm including payload) and the obstacle clearance
  (27\,cm).}
  \label{fig:obstacle}
\end{figure}

\begin{figure}
  \centering
  \includegraphics[width=0.3\textwidth]{figures/evaluation/crawl_series/1.png}
  \includegraphics[width=0.3\textwidth]{figures/evaluation/crawl_series/2.png}
  \includegraphics[width=0.3\textwidth]{figures/evaluation/crawl_series/3.png}
  \caption{The robot detects the obstacle and successfully crawls underneath it.}
  \label{fig:crawl}
\end{figure}
