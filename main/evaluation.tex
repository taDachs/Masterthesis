% !TeX spellcheck = en_US
% !TeX encoding = UTF-8
% !TeX root = ../thesis.tex

\chapter{Evaluation}

\begin{figure}
  \begin{center}
    \includegraphics[width=0.95
    \textwidth]{figures/evaluation/robot_obstacle.jpeg}
    \end{center}
    \caption{Comparison between the robot height and the obstacle height. The robot measures 45cm in
    height (with payload) and the obstacle has a clearance of 27cm.}
  \label{fig:obstacle}
\end{figure}


\begin{figure}
  \begin{center}
    \includegraphics[width=0.3\textwidth]{figures/evaluation/crawl_series/1.png}
    \includegraphics[width=0.3\textwidth]{figures/evaluation/crawl_series/2.png}
    \includegraphics[width=0.3\textwidth]{figures/evaluation/crawl_series/3.png}
  \end{center}
  \caption{The robot is able to detect the obstruction and crawl under it.}
  \label{fig:crawl}
\end{figure}


\begin{figure}
  \begin{center}
    \includegraphics[width=0.95
    \textwidth]{figures/evaluation/terrain_level.png}
    \end{center}
    \caption{}
  \label{fig:terrain-level}
\end{figure}

\begin{figure}
  \begin{center}
    \includegraphics[width=0.95
    \textwidth]{figures/evaluation/qf_loss.png}
    \end{center}
    \caption{}
  \label{fig:qf-loss}
\end{figure}

\begin{figure}
  \begin{center}
    \includegraphics[width=0.95
    \textwidth]{figures/evaluation/actor_loss.png}
    \end{center}
    \caption{}
  \label{fig:actor-loss}
\end{figure}
